\section{Introduction}

Understanding of the wildlife population dynamics is crucial for the conservation of biodiversity. 

Additionally, controlling the population of certain species, especially wild boars that poses significant threats to agriculture and safety of local communities\cite{croftTooManyWild2020,quiros-fernandezHuntersServingEcosystem2017}, is essential for the sustainable development of the region and establishment of heathy relationships between human and wildlife\cite{masseiFertilityControlMitigate2014}. For fertility control or hunting to be effective and ethically necessary, it is essential to have accurate data on the population dynamics of the target species.

Arial footage was considered an important source of data for wildlife monitoring, and companies like DJI has mature solutions for low maintenance arial patrols. However, in the context of forests very common in mountainous regions, the dense canopy leaves only a small fraction of the forest floor visible from the air. 

This makes camera trapping a more suitable solution for monitoring wildlife in these regions. 

This study proposes a tested practical solution ready to be deployed at a large scale.

